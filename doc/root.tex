% interactnlmsample.tex
% v1.05 - August 2017

\documentclass[]{interact}
\usepackage{comment}

\usepackage{epstopdf}% To incorporate .eps illustrations using PDFLaTeX, etc.
\usepackage[caption=false]{subfig}% Support for small, `sub' figures and tables
%\usepackage[nolists,tablesfirst]{endfloat}% To `separate' figures and tables from text if required
%\usepackage[doublespacing]{setspace}% To produce a `double spaced' document if required
%\setlength\parindent{24pt}% To increase paragraph indentation when line spacing is doubled

\usepackage{cite}
\usepackage{algorithm}
\usepackage{algorithmic}
\usepackage{color}
\usepackage{amsmath}
\usepackage{url}

\usepackage[numbers,sort&compress]{natbib}% Citation support using natbib.sty
\bibpunct[, ]{[}{]}{,}{n}{,}{,}% Citation support using natbib.sty
\renewcommand\bibfont{\fontsize{10}{12}\selectfont}% Bibliography support using natbib.sty
\makeatletter% @ becomes a letter
\def\NAT@def@citea{\def\@citea{\NAT@separator}}% Suppress spaces between citations using natbib.sty
\makeatother% @ becomes a symbol again

\theoremstyle{plain}% Theorem-like structures provided by amsthm.sty
\newtheorem{theorem}{Theorem}[section]
\newtheorem{lemma}[theorem]{Lemma}
\newtheorem{corollary}[theorem]{Corollary}
\newtheorem{proposition}[theorem]{Proposition}

\theoremstyle{definition}
\newtheorem{definition}[theorem]{Definition}
\newtheorem{example}[theorem]{Example}

\theoremstyle{remark}
\newtheorem{remark}{Remark}
\newtheorem{notation}{Notation}

\newcommand{\argmax}{\mathop{\rm argmax}\limits}
\newcommand{\argmin}{\mathop{\rm argmin}\limits}

\begin{document}

% \articletype{ARTICLE}% Specify the article type or omit as appropriate

\title{ゼロから学ぶLiDAR-Inertial OdometryとSLAM\\〜フルスクラッチで構築する3次元自己位置推定と地図生成〜}

\author{
\name{赤井 直紀\textsuperscript{a, b}}
\affil{\textsuperscript{a}株式会社LOCT~\textsuperscript{b}名古屋大学}}

\maketitle

% \begin{abstract}
% AAA
% \end{abstract}

% \begin{keywords}
% LiDAR-Inertial Odometry, SLAM,
% \end{keywords}


\section{はじめに}

\subsection{背景知識}

ロボットの自律移動や自動車の自動運転を行うにあたっては,地図を構築し,その地図上で走行している位置を認識する技術,いわゆる{\bf 自己位置推定}(Localization)や{\bf Simultaneous Localization and Mapping}(SLAM)というよばれる技術が重要であるとされています.
従来,これらの技術を用いる場合は,{\bf オドメトリ}(Odometry)と呼ばれる移動量推定を行う枠組みが用いられていました.
自己位置推定やSLAMで用いられている技術を端的に述べると,構築された地図(SLAMの場合はオンラインで構築している地図)と,センサの観測値を照合することで,地図上のどの位置に自分が存在しているかを認識する技術を用いています.
この「センサと地図の照合」を行うにあたり,オドメトリから予測された移動量を用いることで,どの程度移動したかを予測することが可能となり,照合を行う際の探索範囲を限定することができるようになります.
そのため,オドメトリを用いると精度や頑健性を向上させることができます.

しかしオドメトリを用いるとなると,自己位置推定やSLAMで用いられる外界センサ(LiDARやカメラ)以外のセンサが必要になります.
オドメトリシステムを構築する上で最も簡単な方法(システムを構築する手間が最もかからないという意味で)は,{\bf Inertial Measurement Unit}(IMU)を用いることだといえます.
IMUは,センサを原点とした加速度と角速度を計測できるセンサであり,これらの値を積分していくだけで移動量を計算することができます.
しかしIMUの計測値が含む誤差は大きく,単に積分して得られた位置や角度の精度は極めて低く,自己位置推定やSLAMには利用できないことがほとんどです.
そのため,IMUだけを用いてオドメトリシステムを構築することは不可能に近いといえます.

移動ロボットや自動運転の分野で最も広く使われているオドメトリシステムは,エンコーダ等を用いて車輪の回転量を計測し,その結果を積分することで移動量を計算する方法です.
これは{\bf ホイールオドメトリ}(Wheel Odometry)や{\bf デッドレコニング}(Dead Reckoning)と呼ばれます.
デッドレコニングは,タイヤの空転等が発生しない限り,短距離であれば移動量を正確に計測することができます.
ただしデッドレコニングを用るには,外界センサだけでなく,移動体のハードウェアにも大きな変更を加える必要がります.
そのためデッドレコニングは,安易に利用できるシステムとは言い難いです.
またデッドレコニングは,車輪の回転量を計測することが前提のため,基本的に車輪型の移動体にしか適用することができません.

デッドレコニングに頼らない移動量の推定方法として,{\bf ビジュアルオドメトリ}(Visual Odometry)が提案されました.
ビジュアルオドメトリとは,画像から得られる特徴を追跡することで移動量を推定する方法です.
そのためビジュアルオドメトリは,車輪型以外の移動体にも適用することが可能です.
しかし一般に,ビジュアルオドメトリの精度はデッドレコニング程高くはないことが知られています.

ビジュアルオドメトリが提案された後に,LiDARを用いて移動量推定を行う{\bf LiDAR Odometry}(LO)も提案されました.
LOでは,LiDARが計測する点群を逐次的に照合していくことで移動量の推定を行います.
LiDARの距離計測の精度は高いため,このような方法で移動量推定を行うLOの精度は,一般的にデッドレコニングの精度より高くなります.
そのためLOは,様々な用途で使われるようになりました.

しかしLOにも弱点がありました.
LiDAR,特に3D LiDARの計測周期は遅く(一般に10~20~Hz程度),高速な動き,特に回転を含む移動量を推定することは困難でした.
この問題を解決する方法として提案されたのが,LiDARとIMUを融合して移動量推定を行う{\bf LiDAR-Inertial Odometry}(LIO)です.
IMUは高周期(100~Hz以上)で加速度と角速度を計測することができるため,LiDARの計測周期の移動量を補間することができます.
この移動の補間を用いると,高速に移動するLiDARによって歪んでしまったLiDARの計測点群を補正することができるようになります.
またLIOでは,LOでは推定されていなかった状態量(速度やIMUの計測値のバイアス)の推定も行います.
そのため,IMUの計測値の積分も正確に行えるようになるため,移動量の推定をより高精度に行うことが可能になりました.





\subsection{LIOの性能と限界}

LIOのアルゴリズムの進化は目覚ましいものがありましたが,LiDARの性能自体もここ数年で大きく進化しました.
一昔前(著者が研究を始めたのが2011年)では,「LiDARは価格コストが高いため,カメラを用いた手法を提案する」というのが論文等では常套文句でした.
しかし今では,日本円で10万円程度で購入可能な3D LiDARも発売されています.
そして驚くことに,このような価格の3D LiDARでも,360度100~m程度のレンジを計測できるようになっており,LIOを用いて高精度な移動量推定を行うということはかなり一般的になってきました.
そしてLIOを用いるだけでも,小規模な環境であれば十分な精度の点群地図を構築することができるようになりました.
そのため,ドローンのような飛翔体にこのような小型のLiDARを搭載し,高精度な点群地図を生成することも容易に行われるようになってきました.

ただしLIOはあくまでオドメトリシステムであるため,移動量推定しか行いません.
そのため,どれだけ高精度に移動量が推定できたとしても,推定量に誤差(ドリフト)が含まれてしまうため,LIOだけを用いて大規模な環境の地図構築を行うことはできません.
特に大規模でループ(一度通過した地点を再度通過すること)が含まれる環境や条件ですと,整合性の取れた地図が構築できなくなってしまいます(同じものが同じ地点に正しくマッピングされなくなります).
前述したSLAMでは,このようなループが含まれる場合であっても,整合性が取れた地図構築を行うことを目的としています.
すなわち,精度の高い地図を構築したい場合には,SLAMの利用は避けられません.

またLIOはあくまで移動量推定のシステムです.
応用上においては,移動量がわかるだけでは嬉しさがあることは少ないといえ,SLAMで構築した地図上で,どの位置にいるかを知れることの方が恩恵が多いといえます.
例えば工場等でAGVやフォークリフトの位置を管理したい場合などには,LIOの利用だけでは不十分であり,自己位置推定の利用が求められます.
そのため,単に高精度のLIOが利用可能になったというだけでは新たな応用システムを提案することは難しく,LIOに含まれるアルゴリズムを正しく理解し,それをSLAMや自己位置推定にも応用していくことが重要になります.





\subsection{既存手法と本書の立ち位置}

LIOやSLAMを行うオープンソースはすでに多数存在しています.
例えばLIOの有名なオープンソースとしてはLIO-SAMやFAST-LIOが挙げられます.
これらの手法の性能は極めて高く,これらをダウンロードして使用するだけでも,十分な移動量推定を行うことができるといえます.
またLIO-SAMにはSLAMの機能も含まれているため,地図構築を行うこともできます.
またLiDAR SLAMの有名なオープンソースとしては,CartographerやGLIMが挙げられます.
これらの性能も極めて高く,様々な環境で極めて精度の高い点群地図を構築することができます.

しかし多くのソースコードは,機能を多く含むため,どうしても規模が大きくなってしまいます.
そのため,初めてSLAMを学ぼうとする人がこれらを見ても,どこから何を追えば良いかの判断が難しく,結局ダウンロードして使うだけになってしまうことが多いと思います.
本書,および対応するソースコードは,ソフトウェアの構成をとにかくシンプルに実装することに重きをおいています.
開発したソースコードは,LIO,SLAM,自己位置推定の機能を有しており,その主な処理は空行を除いて2000行未満のC++で完結しています.
このC++の中に,スキャンマッチング,LiDARとIMUの融合(ルーズカップリングとタイトカップリング),ループ検知,ポーズグラフの最適化といった必要な処理をすべて実装しています(用語の詳細は後ほど解説します).
また依存ライブラリも極力少なくし,ほぼフルスクラッチでLIOやSLAMを実装できるようになるようにしています.
なお,主な依存ライブラリはSophus(Eigenベース)とnanoflannのみになります(他はパラメーター設定のためにYAMLを用いています).
これらはそれぞれ線形・Lie代数を扱うライブラリと,最近某探索を行うライブラリとなっており,LIOやSLAMの根幹となる部分はすべてフルスクラッチで実装しています.
















\chapter{Mathematical Background}
\label{chap:数学的知識}

\section{Notation}

In this book, we primarily deal with real numbers.
Unless otherwise specified, all numbers are assumed to be real.
The notation is as follows: a scalar is denoted by $a \in \mathbb{R}$, an $N$-dimensional vector by ${\bf a} \in \mathbb{R}^{N}$, and an $N \times M$ matrix by $A \in \mathbb{R}^{N \times M}$.
We also frequently use the identity matrix.
To make the dimension explicit, we attach a subscript; for example, the $N$-dimensional identity matrix is denoted by $I_{N}$.














\section{Jacobian}

In this book, we primarily make use of {\bf optimization}.
Optimization refers to the process of minimizing (or maximizing) a function $f({\bf x})$ defined over a variable ${\bf x} = \left( x_{1}, \cdots, x_{N} \right)^{\top} \in \mathbb{R}^{N}$.
Such a function is often referred to as a cost function in the context of optimization.
The goal is to determine the variable ${\bf x}$ that minimizes (or maximizes) the value of $f$.
There exist various approaches to perform optimization, but a fundamental step is to compute the gradient of the function, which is defined in equation~(\ref{eq:scholar_function_jacobian}).
%
\begin{align}
  \frac{ \partial f({\bf x}) }{ \partial {\bf x} } =
  \left( \begin{matrix}
    \frac{ \partial f({\bf x}) }{ \partial x_{1} } &
    \cdots                                         &
    \frac{ \partial f({\bf x}) }{ \partial x_{N} }
  \end{matrix} \right).
  \label{eq:scholar_function_jacobian}
\end{align}
%
The derivative of a function with respect to its vector argument is referred to as the {\bf Jacobian}.
For a vector-valued function ${\bf f}({\bf x}) = \left( f_{1}({\bf x}), \ldots, f_{M}({\bf x}) \right)^{\top} \in \mathbb{R}^{M}$, the Jacobian can also be defined, as expressed in equation~(\ref{eq:vector_function_jacobian}).
%
\begin{align}
  \frac{ \partial {\bf f} \left( {\bf x} \right) }{ \partial {\bf x} } =
  \left( \begin{matrix}
    \frac{ \partial f_{1} \left( {\bf x} \right) }{ \partial x_{1} } &
    \cdots                                             &
    \frac{ \partial f_{1} \left( {\bf x} \right) }{ \partial x_{N} } \\
    %
    \vdots                                             &
    \ddots                                             &
    \vdots                                             \\
    %
    \frac{ \partial f_{M} \left( {\bf x} \right) }{ \partial x_{1} } &
    \cdots                                             &
    \frac{ \partial f_{M} \left( {\bf x} \right) }{ \partial x_{N} } \\
  \end{matrix} \right).
  \label{eq:vector_function_jacobian}
\end{align}
%
In this book, the Jacobian is denoted by $J$, unless otherwise specified.

The computation of the Jacobian is generally carried out according to the definitions given in equations~(\ref{eq:scholar_function_jacobian}) and~(\ref{eq:vector_function_jacobian}).
However, it can also be derived in a slightly different manner, and in this book we primarily adopt this approach.
Specifically, we approximate the change in the function $f({\bf x})$ resulting from a perturbation $\delta {\bf x}$ using a Taylor expansion.
%
\begin{align}
  f \left( {\bf x} + \delta {\bf x} \right) \simeq
  f \left( {\bf x} \right) + 
  J \delta {\bf x} + 
  \frac{1}{2} \delta {\bf x}^{\top} H \delta {\bf x},
  \label{eq:taylor_expansion_approximation_2nd_order}
\end{align}
%
where $H = \partial^{2} f({\bf x}) / \partial {\bf x}^{2}$, which is referred to as the {\bf Hessian}.
By neglecting the second-order infinitesimal terms and assuming that both sides of equation~(\ref{eq:taylor_expansion_approximation_2nd_order}) are equal, we obtain the following expression.
%
\begin{align}
  f \left( {\bf x} + \delta {\bf x} \right) - f \left( {\bf x} \right) = J \delta {\bf x}
  \label{eq:jacobian_difference}
\end{align}
%
In other words, if the difference between $f({\bf x} + \delta {\bf x})$ and $f({\bf x})$ can be expressed in the form $J \delta {\bf x}$, then the Jacobian can be obtained.

As a simple example, consider $f(x) = x^{2}$.
The Jacobian\footnote{Strictly speaking, since the argument is a scalar, it is usually referred to as the derivative; however, in the one-dimensional case, it can formally be regarded as a Jacobian.} of this function is $\partial x^{2} / \partial x = 2x$.
Nevertheless, let us derive the Jacobian using the method shown in equation~(\ref{eq:jacobian_difference}).
%
\begin{align}
  \begin{split}
    (x + \delta x)^{2} - x^{2}
    = & x^{2} + 2 x \delta x + \delta x^{2} - x^{2}, \\
    = & 2 x \delta x,
  \end{split}
  \label{eq:jacobian_difference_example}
\end{align}
%
where the second-order infinitesimal term was neglected by assuming $\delta x^{2} \simeq 0$.
As shown in equation~(\ref{eq:jacobian_difference_example}), we correctly obtained the Jacobian of $f(x) = x^{2}$.
Using this approach, the Jacobian can be computed without directly differentiating the function\footnote{The Jacobian obtained here is derived as a first-order approximation by applying a Taylor expansion and neglecting higher-order infinitesimal terms. In the limit as $\delta x \to 0$, it coincides with the theoretical derivative, whereas for finite differences it provides a numerical approximation.}.













\section{Gauss-Newton Method}
\label{subsec:gauss-newton_method}

The problems addressed in this book are often reduced to optimization problems.
While there exist various methods for solving such problems, in this book we primarily employ the {\bf Gauss-Newton method}.

To introduce the Gauss-Newton method, we first define the state variable ${\bf x} \in \mathbb{R}^{N}$ and the residual (or residual vector) ${\bf r}({\bf x}) \in \mathbb{R}^{M}$, which depends on this state.
Using multiple residual vectors, we then define the following cost function.
%
\begin{align}
  E \left( {\bf x} \right) = \sum_{i} \| {\bf r}_{i} \left( {\bf x} \right) \|_{2}^{2} \in \mathbb{R},
\end{align}
%
where $\| \cdot \|_{2}^{2}$ denotes the squared Euclidean norm of a vector.
The state that minimizes the cost function is then defined as follows.
%
\begin{align}
  {\bf x}^{*} = \argmin_{ {\bf x} } E \left( {\bf x} \right).
\end{align}
%
This means that we seek a state ${\bf x}^{*}$ within the domain of ${\bf x}$ that minimizes the cost function $E({\bf x})$.
Such a solution is referred to as the {\bf optimal solution}.

To consider optimization using the Gauss-Newton method, we first approximate the residual vector by a first-order Taylor expansion.
%
\begin{align}
  {\bf r} \left( {\bf x} + \delta {\bf x} \right) \simeq {\bf r} \left( {\bf x} \right) + J \delta {\bf x},
  \label{eq:error_vector_taylor}
\end{align}
%
where $J$ denotes the Jacobian of the residual vector ${\bf r}$ with respect to the state vector ${\bf x}$, i.e., $J = \partial {\bf r} / \partial {\bf x} \in \mathbb{R}^{M \times N}$.
Next, we consider the squared norm of the approximated residual vector given in equation~(\ref{eq:error_vector_taylor}).
%
\begin{align}
  \begin{split}
    \left( {\bf r} + J \delta {\bf x} \right)^{\top} \left( {\bf r} + J \delta {\bf x} \right)
    %
    = & \left( {\bf r}^{\top} + \delta {\bf x}^{\top} J^{\top} \right) \left( {\bf r} + J \delta {\bf x} \right), \\
    %
    = & {\bf r}^{\top} {\bf r} + {\bf r}^{\top} J \delta {\bf x} + \delta {\bf x}^{\top} J^{\top} {\bf r} + \delta {\bf x}^{\top} J^{\top} J \delta {\bf x}, \\
    %
    = & {\bf r}^{\top} {\bf r} + 2 \delta {\bf x}^{\top} J^{\top} {\bf r} + \delta {\bf x}^{\top} J^{\top} J \delta {\bf x},
  \end{split}
  \label{eq:error_vector_taylor_sq_norm}
\end{align}
%
where we use the relation ${\bf r}^{\top} J \delta {\bf x} = \delta {\bf x}^{\top} J^{\top} {\bf r}$.

Next, we regard equation~(\ref{eq:error_vector_taylor_sq_norm}) as a function of $\delta {\bf x}$, denoted by $f(\delta {\bf x})$, and take its derivative with respect to $\delta {\bf x}$.
%
\begin{align}
  \frac{ \partial f \left( \delta {\bf x} \right) }{ \partial \delta {\bf x} } = 2 J^{\top} {\bf r} + 2 J^{\top} J \delta {\bf x}.
  \label{eq:error_vector_taylor_sq_norm_partial}
\end{align}
%
Then, by assuming that equation~(\ref{eq:error_vector_taylor_sq_norm_partial}) equals ${\bf 0}$, the following expression holds.
%
\begin{align}
  J^{\top} J \delta {\bf x} = -J^{\top} {\bf r}.
  \label{eq:gauss_newton_update_value}
\end{align}

Here, let us consider $\delta {\bf x}$ that satisfies equation~(\ref{eq:gauss_newton_update_value}).
The function $f(\delta {\bf x})$ represents an approximation of the residual vector when the state is perturbed by $\delta {\bf x}$, and its squared norm is computed.
By differentiating this with respect to $\delta {\bf x}$ and setting the result equal to ${\bf 0}$, we can obtain the value of $\delta {\bf x}$ that minimizes the squared norm of the approximated residual vector.
In other words, updating ${\bf x}$ by this $\delta {\bf x}$ decreases the cost function.

In the derivation of equation~(\ref{eq:error_vector_taylor_sq_norm}), we considered the approximation of a single residual vector.
However, the actual cost function involves the sum of squared norms of multiple residual vectors.
Therefore, it is necessary to approximate the cost function when the state ${\bf x}$ is perturbed by $\delta {\bf x}$, which leads to equation~(\ref{eq:cost_function_taylor}).
%
\begin{align}
  E \left( {\bf x} + \delta {\bf x} \right) \simeq \sum_{i} \left( {\bf r}_{i}^{\top} {\bf r}_{i} + 2 \delta {\bf x}^{\top} J_{i}^{\top} {\bf r}_{i} + \delta {\bf x}^{\top} J_{i}^{\top} J_{i} \delta {\bf x} \right).
  \label{eq:cost_function_taylor}
\end{align}
%
Similarly, by differentiating equation~(\ref{eq:cost_function_taylor}) with respect to $\delta {\bf x}$ and setting the result to ${\bf 0}$, we obtain the following expression.
%
\begin{align}
  \sum_{i} J_{i}^{\top} J_{i} \delta {\bf x} = -\sum_{i} J_{i}^{\top} {\bf r}_{i}.
  \label{eq:gauss_newton_update_sum}
\end{align}
%
Here, for simplicity, we introduce the following variables.
%
\begin{align}
  \begin{gathered}
    H = \sum_{i} J_{i}^{\top} J_{i}, \\
    {\bf b} = \sum_{i} J_{i}^{\top} {\bf r}_{i}.
  \end{gathered}
  \label{eq:hessian_and_gradient_gauss_newton}
\end{align}
%
With this notation, equation~(\ref{eq:gauss_newton_update_sum}) can be written as $H \delta {\bf x} = -{\bf b}$.
Here, $H$ and ${\bf b}$ are often referred to as the Hessian and the gradient, respectively.
In optimization using the Gauss-Newton method, once $\delta {\bf x}$ satisfying $H \delta {\bf x} = -{\bf b}$ is obtained, the state is updated as follows.
%
\begin{align}
  {\bf x} \leftarrow {\bf x} + \delta {\bf x}.
  \label{eq:gauss_newton_update_vector}
\end{align}
%
Note that from $H \delta {\bf x} = -{\bf b}$, one can naturally derive $\delta {\bf x} = -H^{-1} {\bf b}$.
However, since $H \in \mathbb{R}^{N \times N}$, the direct computation of the inverse becomes intractable when $N$ is large.
For this reason, it is uncommon to compute the inverse explicitly, and we instead refer to $\delta {\bf x}$ as ``the solution satisfying $H \delta {\bf x} = -{\bf b}$.''
When $N$ is not large, computing $H^{-1}$ directly does not pose a practical problem.









\section{Robust Kernel}

In performing optimization, multiple residual vectors are defined.
However, if incorrect correspondences (i.e., mismatches) are used to construct residual vectors, they can adversely affect the optimization.
Typically, the norm of residual vectors arising from mismatches tends to be larger than that of residuals from correct correspondences.
Therefore, it is effective to suppress the influence on optimization according to the magnitude of the residual norm.
To achieve this effect, a {\bf robust kernel} can be introduced.

In general, a cost function with a robust kernel is expressed as follows.
%
\begin{align}
  E \left( {\bf x} \right) = \sum_{i} \rho \left( \| {\bf r}_{i} \left( {\bf x} \right) \|_{2}^{2} \right),
  \label{eq:cost_function_robust_kernel}
\end{align}
%
where $\rho(\cdot)$ represents the robust kernel.
Various types of robust kernels exist, but in this book we employ the {\bf Huber loss}.
The Huber loss is defined as follows.
%
\begin{align}
  \rho \left( s \right)
  =
  \begin{cases}
    s                                & {\rm if} ~ s \leq \delta^{2} \\
    2 \delta \sqrt{ s } - \delta^{2} & {\rm otherwise}
  \end{cases},
  \label{eq:huber_loss}
\end{align}
%
where $\delta$ is an arbitrary positive real number.

To consider the Gauss-Newton method with the Huber loss, we evaluate the Huber loss applied to the linear approximation of the residual vector, as shown in equation~(\ref{eq:error_vector_taylor_sq_norm}); that is, ${\bf r}({\bf x} + \delta {\bf x})$ is approximated by ${\bf r}({\bf x}) + J \delta {\bf x}$.
%
\begin{align}
  \rho \left( \left( {\bf r} + J \delta {\bf x} \right)^{\top} \left( {\bf r} + J \delta {\bf x} \right) \right)
  =
  \rho \left( {\bf r}^{\top} {\bf r} + 2 \delta {\bf x}^{\top} J^{\top} {\bf r} + \delta {\bf x}^{\top} J^{\top} J \delta {\bf x} \right).
\end{align}
%
Here, let $s = {\bf r}^{\top} {\bf r} + 2 \delta {\bf x}^{\top} J^{\top} {\bf r} + \delta {\bf x}^{\top} J^{\top} J \delta {\bf x}$.
From equation~(\ref{eq:huber_loss}), when $s \leq \delta^{2}$, the value of $s$ is directly used.
Therefore, we focus on the case where $s$ exceeds $\delta^{2}$, and differentiate the result of substituting the given $s$ into the Huber loss with respect to $\delta {\bf x}$.
%
\begin{align}
  \begin{split}
    \frac{ \partial \rho \left( s \right) }{ \partial \delta {\bf x} }
    = &
    \frac{ \partial \rho \left( s \right) }{ \partial s }
    \frac{ \partial s }{ \partial \delta {\bf x} }, \\
    = &
    \frac{ \delta }{ \sqrt{s} } \left( 2 J^{T} {\bf r} + 2 J^{\top} J \delta {\bf x} \right).
  \end{split}
  \label{eq:huber_loss_diff}
\end{align}
%
Then, by assuming that equation~(\ref{eq:huber_loss_diff}) equals ${\bf 0}$, we obtain the following result.
%
\begin{align}
  \frac{ \delta }{ \sqrt{s} } J^{\top} J \delta {\bf x} = -\frac{ \delta }{ \sqrt{s} } J^{T} {\bf r}.
  \label{eq:gauss_newton_update_value_huber_loss}
\end{align}
%
Compared with equation~(\ref{eq:gauss_newton_update_value}), equation~(\ref{eq:gauss_newton_update_value_huber_loss}) contains the factor $\delta / \sqrt{s}$ on both sides.
Note that this factor can, in principle, be eliminated.
However, since in practice we consider the sum of multiple residual vectors, and the value of $\delta / \sqrt{s}$ differs for each residual vector, we retain it in the formulation.

Then, as shown in equation~(\ref{eq:hessian_and_gradient_gauss_newton}), by defining the Hessian and the gradient while taking all residual vectors into account, we obtain the following.
%
\begin{align}
  \begin{gathered}
    H = \sum_{i} \rho^{\prime} \left( s_{i} \right) J_{i}^{\top} J_{i}, \\
    {\bf b} = \sum_{i} \rho^{\prime} \left( s_{i} \right) J_{i}^{\top} {\bf r}_{i},
  \end{gathered}
  \label{eq:hessian_and_gradient_gauss_newton_huber_loss}
\end{align}
%
where $\rho^{\prime}(s) = \partial \rho(s) / \partial s$, which, from equation~(\ref{eq:huber_loss}), is given as follows.
%
\begin{align}
  \rho^{\prime} \left( s \right)
  =
  \begin{cases}
    1                             & {\rm if} ~ s \leq \delta^{2} \\
    \frac{ \delta }{ \sqrt{ s } } & {\rm otherwise}
  \end{cases}.
  \label{eq:huber_loss_prime}
\end{align}
%

However, since $s = {\bf r}^{\top} {\bf r} + 2 \delta {\bf x}^{\top} J^{\top} {\bf r} + \delta {\bf x}^{\top} J^{\top} J \delta {\bf x}$, $\rho^{\prime}(\cdot)$ directly depends on $\delta {\bf x}$.
As a result, the linear structure of the Gauss-Newton method breaks down, and the update cannot be obtained in the form shown in equation~(\ref{eq:gauss_newton_update_sum}).
To address this, we let $s_{0} = {\bf r}^{\top} {\bf r}$ and $\delta s = 2 \delta {\bf x}^{\top} J^{\top} {\bf r} + \delta {\bf x}^{\top} J^{\top} J \delta {\bf x}$, and approximate $\rho^{\prime}(s) \simeq \rho^{\prime}(s_{0}) + \rho^{\prime \prime}(s_{0}) \delta s$.
Then, by regarding $\rho^{\prime \prime}(s_{0}) \delta s$ as a higher-order infinitesimal term and neglecting it, we approximate $\rho^{\prime}(s) \simeq \rho^{\prime}(s_{0})$.
Consequently, $\rho^{\prime}(s) \simeq \rho^{\prime}(s_{0})$ no longer depends on $\delta {\bf x}$, thereby preserving the linear structure of the Gauss-Newton method.

Finally, we define the weight $w = \rho^{\prime}({\bf r}^{\top} {\bf r})$ and specify the Hessian and gradient as follows.
%
\begin{align}
  \begin{gathered}
    H = \sum_{i} w_{i} J_{i}^{\top} J_{i}, \\
    {\bf b} = \sum_{i} w_{i} J_{i}^{\top} {\bf r}_{i}.
  \end{gathered}
  \label{eq:hessian_and_gradient_gauss_newton_huber_loss_weight}
\end{align}
%
By using these $H$ and ${\bf b}$ to solve for $\delta {\bf x}$ satisfying $H \delta {\bf x} = -{\bf b}$, and updating the state according to equation~(\ref{eq:gauss_newton_update_vector}), optimization with the Gauss-Newton method incorporating the Huber loss can be performed.
Although the derivation of the state update for the Gauss-Newton method with the Huber loss is somewhat more involved, the actual implementation is straightforward: it simply requires computing the function given in equation~(\ref{eq:huber_loss_prime}) as a weight and applying it to the corresponding Hessian and gradient.
In many cases, employing the Huber loss makes the optimization more robust.
















\section{Lie Group and Lie Algebra}

In this book, we frequently make use of {\bf Lie groups} and {\bf Lie algebras}.
While a rigorous explanation of Lie groups and Lie algebras is beyond the scope of this text, throughout this book the term Lie group will specifically refer to {\bf rotation matrices} and {\bf rigid transformation matrices}.

A rotation matrix is defined as follows.
%
\begin{align}
  \{R \in \mathbb{R}^{3 \times 3} | R^{\top} R = I, {\rm det}(R) = 1\}.
  \label{eq:def_SO3}
\end{align}
%
A rotation matrix, also referred to as the Special Orthogonal group in three dimensions (${\rm SO}(3)$), represents rotations in three-dimensional space.
A rigid transformation matrix is defined as follows.
%
\begin{align}
  \left\{ \left( \begin{matrix} R & {\bf t} \\ {\bf 0}^{\top} & 1 \end{matrix} \right) \in \mathbb{R}^{4 \times 4} | R \in {\rm SO}(3), {\bf t} \in \mathbb{R}^{3} \right\}.
  \label{eq:def_SE3}
\end{align}
%
A rigid transformation matrix, also referred to as the Special Euclidean group in three dimensions (${\rm SE}(3)$), represents a pose in three-dimensional space, consisting of both rotation and translation.
In LIO and SLAM, the state is typically represented using ${\rm SO}(3)$ and ${\rm SE}(3)$.
As is evident from equation~(\ref{eq:def_SE3}), the variables that constitute $T$ are ${\bf t}$ and $R$.
Therefore, in this book we often adopt the simplified notation $T = \left( R \mid {\bf t} \right)$.

The advantage of using Lie groups is that rotations can be handled smoothly and in a mathematically natural way, without sudden jumps or discontinuities in their representation.
Although the details are omitted here, such spaces are referred to as {\bf manifolds}.
While the concept may feel somewhat abstract at first, let us consider the simple example of planar rotation, namely an angle $\theta$ in the $xy$-plane.
The angle $\theta$ is usually defined in the range $0 \leq \theta < 2\pi$ (or $-\pi \leq \theta < \pi$).
Although $\theta = 0$ and $\theta = 2\pi$ are numerically different, they represent the same rotational state.
Consequently, using $\theta$ directly as a representation may give the appearance of discontinuity. 
By contrast, Lie groups allow changes in rotation to be represented on a ``seamless space,'' enabling continuous treatment from start to finish.
Moreover, operations such as addition and differentiation can be performed under a unified mathematical framework, ensuring consistency and smoothness.

On the other hand, representing states on Lie groups makes the treatment somewhat less intuitive.
For example, suppose the robot state is represented by a vector ${\bf x}$ in some vector space.
If the robot state changes by $\Delta {\bf x}$, one might intuitively expect the new state to be given by the simple addition ${\bf x} + \Delta {\bf x}$ (noting that, as mentioned earlier, special care must be taken when angles such as $\theta$ are involved due to their defined ranges).
However, as shown in equations~(\ref{eq:def_SO3}) and~(\ref{eq:def_SE3}), ${\rm SO}(3)$ and ${\rm SE}(3)$ are subject to specific constraints, and performing such addition would violate these constraints.

For instance, let the change in pose, including both translation and rotation in three-dimensional space, be defined as $\Delta T \in {\rm SE}(3)$ as follows.
%
\begin{align}
  \Delta T
%
  = \left( \begin{matrix}
      \Delta R       & \Delta {\bf t} \\
      {\bf 0}^{\top} & 1
    \end{matrix} \right).
\end{align}
%
If we consider the simple addition of $T$ and $\Delta T$, it can be written as follows.
%
\begin{align}
  T + \Delta T
%
  = \left( \begin{matrix}
      R + \Delta R       & {\bf t} + \Delta {\bf t} \\
      {\bf 0}^{\top}     & 2
    \end{matrix} \right) \notin {\rm SE}(3).
\end{align}
%
It is clear that $T + \Delta T$ does not satisfy equation~(\ref{eq:def_SE3}), and therefore it does not represent a rigid transformation matrix.
To reflect the change while preserving the constraints of ${\rm SE}(3)$, we compute $T \Delta T$\footnote{Since the result of matrix multiplication depends on whether it is applied from the left or the right, the choice of how $\Delta T$ acts is important. In this book, changes associated with motion are generally modeled using right multiplication.}.
%
\begin{align}
  T \Delta T
%
  = \left( \begin{matrix}
      R \Delta R     & R \Delta {\bf t} + {\bf t} \\
      {\bf 0}^{\top} & 1
    \end{matrix} \right) \in {\rm SE}(3).
  \label{eq:se3_add}
\end{align}
%
Moreover, when the difference between $T_{1}, T_{2} \in {\rm SE}(3)$ is denoted as $\Delta T$, the relation $T_{1} \Delta T = T_{2}$ holds.
Thus, the difference between $T_{1}$ and $T_{2}$ is defined as follows.
%
\begin{align}
  \begin{split}
    \Delta T
%
    & = T_{1}^{-1} T_{2}, \\
%
    & = \left( \begin{matrix}
          R_{1}^{\top} R_{2} & R_{2} ({\bf t}_{2} - {\bf t}_{1}) \\
          {\bf 0}^{\top}     & 1
        \end{matrix} \right) \in {\rm SE}(3).
  \end{split}
  \label{eq:se3_diff}
\end{align}

By using equations~(\ref{eq:se3_add}) and~(\ref{eq:se3_diff}), state changes in the space of ${\rm SE}(3)$ (or ${\rm SO}(3)$) can be expressed.
However, compared to addition and subtraction in vector spaces, these operations are less intuitive and are not in a form to which the Gauss-Newton method, as described in the previous sections, can be directly applied.
To address this, we consider whether states represented on Lie groups can be associated with vectors.
This is made possible by employing the Lie algebra, which is the vector space corresponding to a Lie group.
Specifically, the Lie algebras associated with ${\rm SO}(3)$ and ${\rm SE}(3)$ are denoted by $\mathfrak{so}(3)$ and $\mathfrak{se}(3)$, respectively.
Strictly speaking, $\mathfrak{so}(3)$ and $\mathfrak{se}(3)$ are sets of matrices in $\mathbb{R}^{3 \times 3}$ and $\mathbb{R}^{4 \times 4}$, but since they can be parameterized by three and six independent real numbers, respectively, they can be represented as three- and six-dimensional vectors.
Furthermore, the mappings between ${\rm SO}(3)$ and $\mathfrak{so}(3)$, and between ${\rm SE}(3)$ and $\mathfrak{se}(3)$, are referred to as the {\bf exponential map} and the {\bf logarithm map}, respectively.
%
\begin{align}
  \begin{gathered}
    \exp: \mathfrak{so}(3) \rightarrow {\rm SO}(3), \\
    \log: {\rm SO}(3) \rightarrow \mathfrak{so}(3), \\
  \end{gathered}
  \label{eq:so3_exp_log_maps}
\end{align}
%
\begin{align}
  \begin{gathered}
    \exp: \mathfrak{se}(3) \rightarrow {\rm SO}(3), \\
    \log: {\rm SO}(3) \rightarrow \mathfrak{se}(3). \\
  \end{gathered}
  \label{eq:se3_exp_log_maps}
\end{align}
%
Note that the exponential and logarithm maps shown in equations~(\ref{eq:so3_exp_log_maps}) and~(\ref{eq:se3_exp_log_maps}) are distinct.
However, in this book we do not explicitly specify whether ${\rm SO}(3)$ or ${\rm SE}(3)$ is used in each case; instead, the type of map is inferred from the variable provided as the argument.

By employing Lie algebras, optimization methods such as the Gauss-Newton method can be applied even when the states are represented on Lie groups.
Before discussing optimization using Lie algebras, however, we first provide an explanation of the exponential and logarithm maps.
The derivations of these maps are omitted, and only their computational procedures are presented.













\subsection{Exponential and Logarithm Maps}

For a three-dimensional vector $\boldsymbol{\phi}$, the exponential map shown in equation~(\ref{eq:so3_exp_log_maps}) is defined as follows.
%
\begin{align}
  \begin{gathered}
    \theta = \| \boldsymbol \phi \|_{2}, \\
    %
    \exp \left( \boldsymbol \phi^{^\wedge} \right)
    =
    I_{3} +
    \frac{ \sin \theta }{ \theta } \boldsymbol \phi{^\wedge} +
    \frac{ 1 - \cos \theta }{ \theta^{2} } \left( \boldsymbol \phi{^\wedge} \right)^{2},
  \end{gathered}
  \label{eq:so3_exp_map}
\end{align}
%
where $\left( \cdot \right)^{\wedge}$ denotes the operation that maps a three-dimensional vector to $\mathfrak{so}(3)$ (or a six-dimensional vector to $\mathfrak{se}(3)$).
In the case of a three-dimensional vector, it can also be regarded as the operation that generates a {\bf skew-symmetric matrix}.
%
\begin{align}
  \boldsymbol \phi^{^\wedge}
%
  = \left( \begin{matrix}
      0         & -\phi_{z} & \phi_{y} \\
      \phi_{z}  & 0         & -\phi_{x} \\
      -\phi_{y} & \phi_{x}  & 0
    \end{matrix} \right) \in \mathfrak{so}(3),
  \label{eq:skew_symmetric_matrix}
\end{align}
%
As a brief aside, one property of skew-symmetric matrices is that ${\bf a}^{\wedge} {\bf b} = -{\bf b}^{\wedge} {\bf a}$, where ${\bf a}, {\bf b} \in \mathbb{R}^{3}$.
%
\begin{align}
  \begin{split}
    {\bf a}^{^\wedge} {\bf b}
%
    & = \left( \begin{matrix}
          -a_{z} b_{y} + a_{y} b_{z} \\
           a_{z} b_{x} - a_{x} b_{z} \\
          -a_{y} b_{x} + a_{x} b_{y}
        \end{matrix} \right), \\
%
    & = \left( \begin{matrix}
          0      &  b_{z} & -b_{y} \\
          -b_{z} & 0      & b_{x} \\
          b_{y}  & -b_{x} & 0
        \end{matrix} \right),
        \left( \begin{matrix}
          a_{x} \\
          a_{y} \\
          a_{z}
        \end{matrix} \right) \\
%
     & = -{\bf b}^{\wedge} {\bf a}.
  \end{split}
\end{align}
%
This property will be used later when computing Jacobians in the implementation.

Furthermore, the logarithm map shown in equation~(\ref{eq:so3_exp_log_maps}) is defined as follows.
%
\begin{align}
  \begin{gathered}
    \theta = \arccos \left( \frac{{\rm tr}(R) - 1}{ 2 } \right), \\
    %
    \log( R ) = \frac{ \theta }{ 2 \sin \theta } \left( R - R^{\top} \right).
  \end{gathered}
  \label{eq:so3_log_map}
\end{align}
%
Note that since $\log(R) = \boldsymbol{\phi}^{\wedge} \in \mathfrak{so}(3)$, we define the operation of extracting the three-dimensional vector $\boldsymbol{\phi} = \left( \phi_{x} ~ \phi_{y} ~ \phi_{z} \right)^{\top}$ from this skew-symmetric matrix as follows.
%
\begin{align}
  \boldsymbol \phi = \left( \boldsymbol \phi^{\wedge} \right)^{\vee}.
\end{align}
%
As will be shown later, the operation of extracting a six-dimensional vector from $\mathfrak{se}(3)$ is similarly denoted by $\left( \cdot \right)^{\vee}$.

In equations~(\ref{eq:so3_exp_map}) and~(\ref{eq:so3_log_map}), the parameter $\theta$ appears in the denominator, which makes the computation unstable when $\theta \ll 1$.
Therefore, in the case of $\theta \ll 1$, the computation is approximated as follows.
%
\begin{align}
  \begin{gathered}
    \exp \left( \boldsymbol \phi^{\wedge} \right) \simeq I_{3} + \boldsymbol \phi^{\wedge} + \frac{1}{2} \left( \boldsymbol \phi^{\wedge} \right)^{2}, \\
%
    \log \left( R \right) \simeq \frac{1}{2} \left( R - R^{\top} \right).
  \end{gathered}
  \label{eq:so3_exp_log_maps_approx}
\end{align}
%

Next, in order to consider the exponential and logarithm maps for ${\rm SE}(3)$, we define the six-dimensional vector $\boldsymbol{\xi} = \left( {\bf v}^{\top} ~ \boldsymbol{\phi}^{\top} \right)^{\top}$.
We then define the operation that maps this six-dimensional vector to $\mathfrak{se}(3)$ as follows\footnote{In equation~(\ref{eq:skew_symmetric_matrix}), $\left( \cdot \right)^{\wedge}$ is defined as the operation that generates a skew-symmetric matrix corresponding to a three-dimensional vector. In the case of a six-dimensional vector, however, it is treated as the operation that generates an element of $\mathfrak{se}(3)$, as shown in equation~(\ref{eq:se3_wedge}).}.
%
\begin{align}
  \boldsymbol \xi^{\wedge} = \left( \begin{matrix}
    \boldsymbol \phi^{\wedge} & {\bf v} \\
    {\bf 0}^{\top}            & 0
  \end{matrix} \right) \in \mathfrak{se}(3),
  \label{eq:se3_wedge}
\end{align}
%
where $\boldsymbol{\phi}^{\wedge}$ denotes the operation that returns the skew-symmetric matrix defined in equation~(\ref{eq:skew_symmetric_matrix}).
Using equation~(\ref{eq:se3_wedge}), the exponential map shown in equation~(\ref{eq:se3_exp_log_maps}) is defined as follows.
%
\begin{align}
  \exp \left( \boldsymbol \xi^{\wedge} \right) = \left( \begin{matrix}
    \exp \left( \boldsymbol \phi^{\wedge} \right) & J_{l} \left( \boldsymbol \phi^{\wedge} \right) {\bf v} \\
    {\bf 0}^{\top}                                & 1
  \end{matrix} \right),
  \label{eq:se3_exp_map}
\end{align}
%
where $\exp\left( \boldsymbol{\phi}^{\wedge} \right)$ corresponds to the expression given in equation~(\ref{eq:so3_exp_map}), and $J_{l}(\cdot)$ denotes the left Jacobian, which is defined as follows.
%
\begin{align}
  J_{l} \left( \boldsymbol \phi^{\wedge} \right)
  =
  I_{3} +
  \frac{ 1 - \cos \theta }{ \theta^{2} } \boldsymbol \phi^{\wedge} +
  \frac{ \theta - \sin \theta }{ \theta^{3} } \left( \boldsymbol \phi^{\wedge} \right)^{2},
  \label{eq:so3_left_jacobian}
\end{align}
%
where $\theta = \| \boldsymbol{\phi} \|_{2}$.

The logarithm map shown in equation~(\ref{eq:se3_exp_log_maps}) is defined as follows.
%
\begin{align}
  \log \left( T \right) = \left( \begin{matrix}
    \boldsymbol \phi^{\wedge} & J_{l}^{-1} \left( \boldsymbol \phi \right) {\bf t} \\
    {\bf 0}^{\top}            & 0
  \end{matrix} \right),
  \label{eq:se3_log_map}
\end{align}
%
where $J_{l}^{-1}(\cdot)$ denotes the inverse of the left Jacobian, which is defined as follows.
%
\begin{align}
  J_{l}^{-1} \left( \boldsymbol \phi \right)
  =
  I_{3} -
  \frac{1}{2} \boldsymbol \phi^{\wedge} + 
  \left( \frac{1}{ \theta^{2} } - \frac{ 1 + \cos \theta }{ 2 \theta \sin \theta } \right) \left( \boldsymbol \phi^{\wedge} \right)^{2}.
  \label{eq:so3_left_jacobian_inverse}
\end{align}
%
Similarly, $\theta = \| \boldsymbol{\phi} \|_{2}$.
Note that since $\log(T) \in \mathfrak{se}(3)$ is a matrix in $\mathbb{R}^{4 \times 4}$, we define the operation of extracting the six-dimensional vector $\boldsymbol{\xi}$ from it as follows.
%
\begin{align}
  \begin{split}
    \boldsymbol \xi
%
    = & \left( \begin{matrix}
      J_{l}^{-1} \left( \boldsymbol \phi \right) {\bf t} \\
      \boldsymbol \phi
    \end{matrix} \right), \\
%
    = & \left( \log \left( T \right) \right)^{\vee},
  \end{split}
\end{align}
%
where $\boldsymbol{\phi} = \left( \log(R) \right)^{\vee}$.

Note that in equations~(\ref{eq:so3_left_jacobian}) and~(\ref{eq:so3_left_jacobian_inverse}), the parameter $\theta$ also appears in the denominator.
Therefore, when $\theta \ll 1$, it is necessary to perform approximate computations, as shown in equation~(\ref{eq:so3_exp_log_maps_approx}), in order to avoid numerical instability.











\subsection{State Update via Lie Algebra}

As described above, the logarithm map enables mapping elements of a Lie group to elements of its Lie algebra, thereby allowing optimization to be applied in a vector space.
This makes it possible, for example, to apply nonlinear optimization algorithms such as the Gauss-Newton method to Lie groups.
However, since the final state must be represented as an element of the Lie group, we summarize here the procedure for state update via Lie algebra.
In the following, we focus on ${\rm SE}(3)$, but the same idea can also be applied to ${\rm SO}(3)$.

For instance, suppose that by using the Gauss-Newton method or a similar approach, an update $\delta \boldsymbol{\xi} \in \mathbb{R}^{6}$ on the Lie algebra corresponding to the state $T \in {\rm SE}(3)$ is obtained.
This update is then mapped back to the Lie group via the exponential map, and the actual state update is performed as follows.
%
\begin{align}
  T \leftarrow \exp \left( \delta \boldsymbol \xi^{\wedge} \right) T.
  \label{eq:se3_update_left}
\end{align}
%
This operation can also be regarded as adding the Lie algebra element $\delta \boldsymbol{\xi}$ to the Lie group element $T$, and in correspondence with equation~(\ref{eq:gauss_newton_update_vector}), it is sometimes written as follows.
%
\begin{align}
  T \leftarrow T \boxplus \delta \boldsymbol \xi.
  \label{eq:se3_update_left_boxplus}
\end{align}
%
In equation~(\ref{eq:se3_update_left}), the state is updated by multiplying $\exp \left( \delta \boldsymbol{\xi}^{\wedge} \right)$ from the left.
This is because, when computing the gradient in optimization, we consider the change in the residual with respect to perturbations applied from the left on the Lie group.
If perturbations from the right are considered instead, $\exp \left( \delta \boldsymbol{\xi}^{\wedge} \right)$ must be multiplied from the right, and care must be taken in this case.

Assuming that $\boldsymbol{\xi} = \left( \log(T) \right)^{\vee}$, the updated state vector on the Lie algebra becomes $\boldsymbol{\xi} + \delta \boldsymbol{\xi}$.
If $\delta \boldsymbol{\xi}$ is regarded as an infinitesimal perturbation, the {\bf Baker-Campbell-Hausdorff} (BCH) expansion leads to the following approximation.
%
\begin{align}
  \begin{split}
    \exp\left( (\boldsymbol \xi + \delta \boldsymbol \xi)^{\wedge} \right)
    \simeq & 
    \exp(\delta \boldsymbol \xi^{\wedge}) \exp(\boldsymbol \xi^{\wedge}), \\
    = &
    \exp(\delta \boldsymbol \xi^{\wedge}) T.
  \end{split}
\end{align}
%
Therefore, it can be confirmed that the state update can be performed as shown in equation~(\ref{eq:se3_update_left}).












\subsection{Jacobian Computation Using Lie Groups}

As described in Section~\ref{subsec:gauss-newton_method}, deriving the Jacobian is essential when applying the Gauss-Newton method.
For example, suppose we have a residual vector ${\bf r} \in \mathbb{R}^{N}$ that depends on $T \in {\rm SE}(3)$.
To obtain the Jacobian, it suffices to determine $J$ that satisfies the following equation.
%
\begin{align}
  {\bf r} \left( \exp \left( \delta \boldsymbol \xi^{\wedge} \right) T \right) - {\bf r}(T)
= J \delta \boldsymbol \xi.
\end{align}
%
Note that $J \in \mathbb{R}^{N \times 6}$ (or $\mathbb{R}^{N \times 3}$ in the case of ${\bf r}(R) \in \mathbb{R}^{N}, ; R \in {\rm SO}(3)$).
Examples of defining residual vectors and deriving their Jacobians are presented in the following chapters.

In differentiating residual vectors, it is sometimes necessary to compute derivatives between elements of Lie groups.
To illustrate this, let us first consider the simple case of $\partial T / \partial T$.
In this case, it suffices to determine the Jacobian $J$ that satisfies the following equation.
%
\begin{align}
  T^{-1} \left( \exp \left( \delta \boldsymbol \xi^{\wedge} \right) T \right) = I_{4} + \left( J \delta \boldsymbol \xi \right)^{\wedge}.
  \label{eq:dT_dT}
\end{align}
%
This corresponds to considering the difference between $\exp \left( \delta \boldsymbol{\xi}^{\wedge} \right) T$ and $T$, as shown in equation~(\ref{eq:se3_diff}).
In particular, since $T^{-1} T = I_{4}$, this amounts to analyzing infinitesimal variations around the identity element of the Lie group.
Therefore, this is equivalent to considering the expression $f({\bf x} + \delta {\bf x}) - f({\bf x}) = J \delta {\bf x}$ in a vector space.
Note, however, that $J \delta \boldsymbol{\xi}$ is a six-dimensional vector, and thus it cannot be directly added to $I_{4}$.
Instead, it is mapped to the corresponding element of $\mathfrak{se}(3)$ through the $\left( \cdot \right)^{\wedge}$ operator.
Then, $T^{-1} \exp \left( \delta \boldsymbol{\xi}^{\wedge} \right) T$ can be written as follows.
%
\begin{align}
  T^{-1} \exp \left( \delta \boldsymbol \xi^{\wedge} \right) T = \exp \left( \left( \operatorname{Ad}_{T^{-1}} \delta \boldsymbol \xi \right)^{\wedge} \right),
\end{align}
%
where $\operatorname{Ad}_{T} \in \mathbb{R}^{6 \times 6}$ is referred to as the {\bf adjoint} operator (its detailed definition will be given in the next subsection).
Considering that $\delta \boldsymbol{\xi}$ represents an infinitesimal perturbation, we can approximate it as follows.
%
\begin{align}
  \exp \left( \left( \operatorname{Ad}_{T^{-1}} \delta \boldsymbol \xi \right)^{\wedge} \right) \simeq I_{4} + \left( \operatorname{Ad}_{T^{-1}} \delta \boldsymbol \xi \right)^{\wedge}.
  \label{eq:dT_dT_Ad}
\end{align}
%
By comparing equations~(\ref{eq:dT_dT}) and~(\ref{eq:dT_dT_Ad}), we find that $J = \operatorname{Ad}_{T^{-1}}$, which gives the result of $\partial T / \partial T$.







\subsection{Adjoint}

The adjoint operator is defined as the one that satisfies the following equation.
%
\begin{align}
  \begin{gathered}
    \left( \operatorname{Ad}_{R} \boldsymbol \phi \right)^{\wedge} = R \boldsymbol \phi^{\wedge} R^{\top}, \\
%
    \left( \operatorname{Ad}_{T} \boldsymbol \xi \right)^{\wedge} = T \boldsymbol \xi^{\wedge} T^{-1},
  \end{gathered}
  \label{eq:adjoint}
\end{align}
%
where $\boldsymbol{\phi} \in \mathbb{R}^{3}$ and $\boldsymbol{\xi} \in \mathbb{R}^{6}$.
Although the detailed derivations are omitted, the adjoint operators are defined as follows.
%
\begin{align}
  \operatorname{Ad}_{R} = R,
  \label{eq:adjoint_so3}
\end{align}
%
\begin{align}
  \operatorname{Ad}_{T} = \left( \begin{matrix}
    R & {\bf t}^{\wedge} R \\
    0 & R
  \end{matrix} \right).
  \label{eq:adjoint_se3}
\end{align}







\section{Coordinate Systems and Their Notation}

\begin{figure}[!t]
  \centering
  \includegraphics[width=0.4\textwidth]{../figs/frames.pdf}
  \caption{Coordinate frames employed in this work.}
  \label{fig:frames}
\end{figure}

In this book, the coordinate systems considered are primarily the four shown in Fig.~\ref{fig:frames}: the map, odometry, IMU, and LiDAR frames.
Each frame is denoted by the subscripts $M$, $O$, $I$, and $L$, respectively.
For example, when a point ${\bf p}$ is expressed in each of these frames, we place the corresponding superscript in the upper left and denote them as ${}^{M}{\bf p}$, ${}^{O}{\bf p}$, ${}^{I}{\bf p}$, and ${}^{L}{\bf p}$, respectively.
When considering a coordinate transformation (from a Source $S$ to a Target $T$), it is expressed using either of the following notations.
%
\begin{align}
  \begin{gathered}
    {}^{T}{\bf p} = {}^{T}R_{S} {}^{S}{\bf p} + {}^{T}{\bf t}_{S}, \\
%
    {}^{T}{\bf p} = {}^{T}T_{S} {}^{S}{\bf p}.
  \end{gathered}
  \label{eq:point_transformation}
\end{align}
%
Note that when using ${\rm SE}(3)$, we have ${\bf p} \in \mathbb{R}^{4}$, with the fourth element set to $1$.

For example, when transforming a point from the IMU frame to the odometry frame, it is expressed as ${}^{O}{\bf p} = {}^{O}R_{I} {}^{I}{\bf p} + {}^{O}{\bf t}_{I}$, or equivalently as ${}^{O}{\bf p} = {}^{O}T_{I} {}^{I}{\bf p}$.
Similarly, when considering the direct transformation from the IMU frame to the map frame, either of the following notations can be used.
%
\begin{align}
  \begin{gathered}
    {}^{M}{\bf p} = {}^{M}R_{O} \left( {}^{O}R_{I} {}^{I}{\bf p} + {}^{O}{\bf t}_{I} \right) + {}^{M}{\bf t}_{O}, \\
%
    {}^{M}{\bf p} = {}^{M}T_{O} {}^{O}T_{I} {}^{I}{\bf p}.
  \end{gathered}
  \label{eq:point_transformation_synthesis}
\end{align}
%
Here, let ${}^{M}T_{I} = {}^{M}T_{O} {}^{O}T_{I}$, and denote the corresponding translation vector and rotation matrix as ${}^{M}{\bf t}_{I}$ and ${}^{M}R_{I}$, respectively.
With this, equation~(\ref{eq:point_transformation_synthesis}) can be rewritten as a single transformation, as shown in equation~(\ref{eq:point_transformation}).

As for the coordinate transformations illustrated in Fig.~\ref{fig:frames}, in general, the transformation between the IMU and LiDAR, ${}^{I}T_{L}$, is static and thus obtained in advance through calibration.
LIO then estimates ${}^{O}T_{I}$, while SLAM (or localization) estimates ${}^{M}T_{O}$.
However, ${}^{I}T_{L}$ can also be optimized online.
Moreover, in some cases, one may assume that LIO directly estimates the transformation to the map frame, without relying on SLAM or localization.
Therefore, the coordinate relations shown in Fig.~\ref{fig:frames} are not necessarily strict requirements.



\section{スキャンマッチング}

\subsection{概要}

スキャンマッチングとは,2つの点群を正しく照合できるような剛体変換$T \in \mathrm{SE}(3)$を求めることです.
ここで$T$は,回転行列$R \in \mathrm{SO}(3)$と並進ベクトル${\bf t} \in \mathbb{R}^{3}$を含みます.

今,点群$\mathcal{P} = ({\bf p}_{1}, ..., {\bf p}_{N})$と$\mathcal{Q} = ({\bf q}_{1}, ..., {\bf q}_{M})$を照合させることを考えます.
ただし${\bf p}, {\bf q} \in \mathbb{R}^{3}$です.
このとき,以下のコスト関数を考えます.
%
\begin{align}
  E = \sum_{i=1}^{N} \left\| {\bf q}_{i} - \left( R {\bf p}_{i} + {\bf t} \right) \right\|_{2}^{2}
  \label{eq:icp_cost_SO(3)}
\end{align}
%
ここで${\bf q}_{i}$は,${\bf p}_{i}$を剛体変換した点$R {\bf p}_{i} + {\bf t}$に最も近い$\mathcal{Q}$内の点です.
式(\ref{eq:icp_cost_SO(3)})は,以下のように書くことも可能です.
%
\begin{align}
  E = \sum_{i=1}^{N} \left\| {\bf q}_{i} - T {\bf p}_{i} \right\|_{2}^{2}
  \label{eq:icp_cost_SE(3)}
\end{align}
%
なおこの場合,${\bf p}, {\bf q} \in \mathbb{R}^{4}$となり,それぞれの4要素目には1が入ることになります.
そのため,$T {\bf p}$も4要素目が1の4次元ベクトルになりますが,${\bf q}$の4要素目も1となるため,${\bf q} - T {\bf p}$の4要素目は常に0になることととなり,結果としてコスト関数の値は式(\ref{eq:icp_cost_SO(3)})に示す値と同じになります.
なお以下では,${\bf q} - T {\bf p}$を誤差ベクトル${\bf e}$として定めます.

スキャンマッチングでは,以下に示す剛体変換を求めることを考えます.
%
\begin{align}
  T^{*} = \argmin_{T} E
  \label{eq:icp_scan_matching}
\end{align}
%
式(\ref{eq:icp_scan_matching})は,コスト関数$E$を最小化する姿勢$T^{*}$を求めるという意味になります.
この$T^{*}$は,一般的に反復処理を行うことで求めるられるため,Iterative Closest Points(ICP)スキャンマッチングとも呼ばれます.
なお式(\ref{eq:icp_cost_SE(3)})に示すコストを最小にするスキャンマッチングは,対応する点同士の距離を最小にするため,point-to-point ICPとも呼ばれます.

point-to-point ICPは一般的にノイズに脆弱であるといわれています.
そのため本書では,より頑健性の高い点と面の距離を最小化するpoint-to-plane ICPについて考えます.
point-to-plane ICPでは,以下のコスト関数を考えます.
%
\begin{align}
  E = \sum_{i=1}^{N} \left( {\bf n}_{i}^{\top} {\bf e}_{i} \right)^{2}
  \label{eq:point-to-plane_icp_cost_SE(3)}
\end{align}
%
ここで${\bf n}_{i}^{\top}$は,${\bf q}_{i}$の周辺の点を用いて計算した面の3次元空間内での法線ベクトルです.
なお,${\bf e}$が4次元ベクトルであるため${\bf n}$も4次元ベクトルとなりますが,${\bf e}$の4要素目は常に0であるため,${\bf n}$の4要素目がいくつであっても計算に違いは表れません.





\subsection{ヤコビアンの計算}

本書では,式(\ref{eq:point-to-plane_icp_cost_SE(3)})に示すコスト関数の最小化を行うために,ガウス・ニュートン法を用います.
そのために,残差$r = {\bf n}^{\top} {\bf e}$の姿勢$T$に関するヤコビアンを求めます.
このヤコビアンは,連鎖則を用いて以下のように計算できます.
%
\begin{align}
  \frac{ \partial r }{ \partial T } = \frac{ \partial r }{ \partial {\bf e} }
                                      \frac{ \partial {\bf e} }{ \partial T }
\end{align}
%
ここで,$\frac{ \partial r }{ \partial {\bf e} }$は明らかに${\bf n}^{\top}$です.
そのため,$\frac{ \partial {\bf e} }{ \partial T }$についてのみ詳細の計算方法を示します.

残差ベクトル${\bf e}$は,明らかに姿勢$T$に関する関数になっています.
そのため,次の微小変化を考え,ることでヤコビアン$J$を導出します.
%
\begin{align}
  {\bf e}(T \oplus \delta T) - {\bf e}(T) = 
  {\bf e}(\exp(\delta \boldsymbol \xi) T) - {\bf e}(T) \simeq
  J \delta \boldsymbol \xi
\end{align}
%
$\delta \boldsymbol \xi^{\top} = \left( \delta {\bf t}^{\top} ~ \delta \boldsymbol \theta^{\top} \right)^{\top} \in \mathfrak{se}(3)$
%
\begin{align}
  \begin{split}
    & {\bf q} - \exp(\delta \boldsymbol \xi) T {\bf p} - \left( {\bf q} - T {\bf p} \right) \\
    %
    = & - \left( \exp(\delta \boldsymbol \xi) - I_{4} \right) T {\bf p} \\
    %
    = & - \left( \left( \begin{matrix} I_{3} + [\delta \boldsymbol \theta]_{\times} & \delta {\bf t} \\ {\bf 0}^{\top} & 1 \end{matrix} \right) - I_{4} \right) \left( \begin{matrix} R {\bf p} + {\bf t} \\ 1 \end{matrix} \right) \\
    %
    = & - \left( \begin{matrix} [\delta \boldsymbol \theta]_{\times} & \delta {\bf t} \\ {\bf 0}^{\top} & 0 \end{matrix} \right) \left( \begin{matrix} {\bf p}' \\ 1 \end{matrix} \right) \\
    %
     = & - \left( \begin{matrix} [\delta \boldsymbol \theta]_{\times} {\bf p}' & \delta {\bf t} \\ {\bf 0}^{\top} & 0 \end{matrix} \right) \\
    %
     = & \left( \begin{matrix} [{\bf p}']_{\times} \delta \boldsymbol \theta & - \delta {\bf t} \\ {\bf 0}^{\top} & 0 \end{matrix} \right) \\
     %
     = & \left( \begin{matrix} -I_{3} & [{\bf p}']_{\times} \\ {\bf 0}^{\top} & {\bf 0}^{\top} \end{matrix} \right) \left( \begin{matrix} \delta {\bf t} \\ \delta \boldsymbol \theta \end{matrix} \right) =
     J \delta \boldsymbol \xi
  \end{split}
\end{align}
%
なお${\bf p}' = R {\bf p} + {\bf t}$と置き,$[\delta \boldsymbol \theta]_{\times} {\bf p}' = -[{\bf p}']_{\times} \delta \boldsymbol \theta$を用いました.
よって,残差に対するヤコビアンは以下となります.
%
\begin{align}
  \begin{split}
    \frac{ \partial r }{ \partial T } = & {\bf n}^{\top} \left( \begin{matrix} -I_{3} & [{\bf p}']_{\times} \\ {\bf 0}^{\top} & {\bf 0}^{\top} \end{matrix} \right) \\
    %
    = & \left( \begin{matrix} -{\bf n}^{\top} & {\bf n}^{\top} [{\bf p}']_{\times} \end{matrix} \right) \in \mathbb{R}^{1 \times 6}
  \end{split}
\end{align}
%
ただし最後の${\bf n}^{\top}$は3次元の法線ベクトルになります.

%
\begin{align}
  \begin{gathered}
    H = \sum_{i=1}^{N} J_{i}^{\top} J_{i} \\
    {\bf b} = \sum_{i=1}^{N} J_{i}^{\top} r_{i}
  \end{gathered}
\end{align}





\cite{BorensteinJRS1997}.








%%%%%%%%%%%%%%%%%%%%%%%%%%%%%%%%%%%%%%%%%%%%%%%%%%%%%%%%%%%%%%%%%%%%%%%%%%%%%%%%





















% \addtolength{\textheight}{-7.6cm}   % This command serves to balance the column lengths
                                  % on the last page of the document manually. It shortens
                                  % the textheight of the last page by a suitable amount.
                                  % This command does not take effect until the next page
                                  % so it should come on the page before the last. Make
                                  % sure that you do not shorten the textheight too much.

%%%%%%%%%%%%%%%%%%%%%%%%%%%%%%%%%%%%%%%%%%%%%%%%%%%%%%%%%%%%%%%%%%%%%%%%%%%%%%%%



%%%%%%%%%%%%%%%%%%%%%%%%%%%%%%%%%%%%%%%%%%%%%%%%%%%%%%%%%%%%%%%%%%%%%%%%%%%%%%%%



%%%%%%%%%%%%%%%%%%%%%%%%%%%%%%%%%%%%%%%%%%%%%%%%%%%%%%%%%%%%%%%%%%%%%%%%%%%%%%%%
%\section*{APPENDIX}
%
%Appendixes should appear before the acknowledgment.

\section*{ACKNOWLEDGMENT}

This work was supported by KAKENHI under Grant 23K03773.
% \balance
\bibliographystyle{unsrt}
\bibliography{root.bib}


\end{document}
